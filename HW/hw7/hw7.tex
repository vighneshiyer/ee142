%texexptitled======================================================================
% lab1-gcd
%-----------------------------------------------------------------------
%

\documentclass[11pt]{article}

% Package includes

\usepackage{graphicx}
\usepackage{color}
\usepackage{comment}
\usepackage{multirow}
\usepackage{askmaps}
\usepackage{amssymb}
\usepackage{amsmath}
\usepackage{tikz}
\usepackage{circuitikzgit}
\usetikzlibrary{arrows, positioning, shapes.geometric, circuits.logic.US}
\tikzstyle{line}=[draw]
\tikzstyle{arrow}=[draw, -latex]

% Wrap long URLs with hyphens
\PassOptionsToPackage{hyphens}{url}\usepackage{hyperref}
\usepackage{pdftexcmds}
\usepackage{upquote}
\usepackage{textcomp}
\usepackage{minted}
\usepackage[listings]{tcolorbox}
\usepackage{enumerate}
\usepackage{enumitem}
\usepackage{mathtools}
\DeclarePairedDelimiter{\ceil}{\Big\lceil}{\Big\rceil}

\tcbset{
texexp/.style={colframe=black, colback=lightgray!15,
         coltitle=white,
         fonttitle=\small\sffamily\bfseries, fontupper=\small, fontlower=\small},
     example/.style 2 args={texexp,
title={Question \thetcbcounter: #1},label={#2}},
}

\newtcolorbox{texexp}[1]{texexp}
\newtcolorbox[auto counter]{texexptitled}[3][]{%
example={#2}{#3},#1}

\setlength{\topmargin}{-0.5in}
\setlength{\textheight}{9in}
\setlength{\oddsidemargin}{0in}
\setlength{\evensidemargin}{0in}
\setlength{\textwidth}{6.5in}

% Useful macros

\newcommand{\note}[1]{{\bf [ NOTE: #1 ]}}
\newcommand{\fixme}[1]{{\bf [ FIXME: #1 ]}}
\newcommand{\wunits}[2]{\mbox{#1\,#2}}
\newcommand{\um}{\mbox{$\mu$m}}
\newcommand{\xum}[1]{\wunits{#1}{\um}}
\newcommand{\by}[2]{\mbox{#1$\times$#2}}
\newcommand{\byby}[3]{\mbox{#1$\times$#2$\times$#3}}


\newenvironment{tightlist}
{\begin{itemize}
 \setlength{\parsep}{0pt}
 \setlength{\itemsep}{-2pt}}
{\end{itemize}}

\newenvironment{titledtightlist}[1]
{\noindent
 ~~\textbf{#1}
 \begin{itemize}
 \setlength{\parsep}{0pt}
 \setlength{\itemsep}{-2pt}}
{\end{itemize}}

% Change spacing before and after section headers

\makeatletter
\renewcommand{\section}
{\@startsection {section}{1}{0pt}
 {-2ex}
 {1ex}
 {\bfseries\Large}}
\makeatother

\makeatletter
\renewcommand{\subsection}
{\@startsection {subsection}{1}{0pt}
 {-1ex}
 {0.5ex}
 {\bfseries\normalsize}}
\makeatother

% Reduce likelihood of a single line at the top/bottom of page

\clubpenalty=2000
\widowpenalty=2000

% Other commands and parameters

\pagestyle{myheadings}
\setlength{\parindent}{0in}
\setlength{\parskip}{10pt}

% Commands for register format figures.

\newcommand{\instbit}[1]{\mbox{\scriptsize #1}}
\newcommand{\instbitrange}[2]{\instbit{#1} \hfill \instbit{#2}}

\graphicspath{{./figs/}}


%-----------------------------------------------------------------------
% Document
%-----------------------------------------------------------------------

\begin{document}
\def\PYZsq{\textquotesingle}


\newcommand{\headertext}{EE142 Problem Set 7}
\renewcommand{\thesubsection}{\thesection.\alph{subsection}}

\title{\vspace{-0.4in}\Large \bf \headertext \vspace{-0.1in}}
\author{Vighnesh Iyer}

\date{\today}
\maketitle

\markboth{\headertext}{\headertext}
\thispagestyle{empty}

\section{Noise Figure of Cascade Blocks and Lossy Transmission Line}
\begin{enumerate}[label=(\alph*)]
    \begin{figure}[H]
        \centering \includegraphics[width=\textwidth-2cm]{problem1a_schematic.png}
    \end{figure}

    \item {\color{blue}For the above two cascade circuits, calculate the power gains and noise figures for each stage (i.e. $S_{21,1}$, $S_{21,2}$, $NF_1$, $NF_2$) and the two stage circuits ($S_{21,total}$, $NF_{total}$).
    The resistors are assumed to be noiseless.}

    \subsection{Cascade 1}

    For the first cascase's stage 1, we begin by input referring the noise sources and collapsing the voltage and current noise into $\overline{v_{eq}^2}$.
    From lecture:

    \begin{align*}
        \overline{v_{eq}^2} &= \overline{v_n^2} + \overline{i_n^2} R_s^2 \\
        F &= 1 + \frac{N_{amp,i}}{N_s} = 1 + \frac{\overline{v_{eq}^2}}{\overline{v_s^2}}
    \end{align*}

    Assume we are calculating noise figure in a $50 \Omega$ environment, $R_s = 50 \Omega$ and $\overline{v_s^2} = 4 kT R_s$.
    We assume that all noise sources represented are defined as \emph{spot noise}.

    \begin{align*}
        F_1 = 1 + \frac{\overline{v_{eq,1}^2}}{\overline{v_s^2}} = \frac{4kT\cdot 20 + 4kT \cdot 2 \cdot 50^2}{4kT \cdot R_s}
    \end{align*}

\end{enumerate}
\end{document}
