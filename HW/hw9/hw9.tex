%texexptitled======================================================================
% lab1-gcd
%-----------------------------------------------------------------------
%

\documentclass[11pt]{article}

% Package includes

\usepackage{graphicx}
\usepackage{color}
\usepackage{comment}
\usepackage{multirow}
\usepackage{askmaps}
\usepackage{amssymb}
\usepackage{amsmath}
\usepackage{tikz}
\usepackage{circuitikzgit}
\usetikzlibrary{arrows, positioning, shapes.geometric, circuits.logic.US}
\tikzstyle{line}=[draw]
\tikzstyle{arrow}=[draw, -latex]

% Wrap long URLs with hyphens
\PassOptionsToPackage{hyphens}{url}\usepackage{hyperref}
\usepackage{pdftexcmds}
\usepackage{upquote}
\usepackage{textcomp}
\usepackage{minted}
\usepackage[listings]{tcolorbox}
\usepackage{enumerate}
\usepackage{enumitem}
\usepackage{mathtools}
\DeclarePairedDelimiter{\ceil}{\Big\lceil}{\Big\rceil}

\tcbset{
texexp/.style={colframe=black, colback=lightgray!15,
         coltitle=white,
         fonttitle=\small\sffamily\bfseries, fontupper=\small, fontlower=\small},
     example/.style 2 args={texexp,
title={Question \thetcbcounter: #1},label={#2}},
}

\newtcolorbox{texexp}[1]{texexp}
\newtcolorbox[auto counter]{texexptitled}[3][]{%
example={#2}{#3},#1}

\setlength{\topmargin}{-0.5in}
\setlength{\textheight}{9in}
\setlength{\oddsidemargin}{0in}
\setlength{\evensidemargin}{0in}
\setlength{\textwidth}{6.5in}

% Useful macros

\newcommand{\note}[1]{{\bf [ NOTE: #1 ]}}
\newcommand{\fixme}[1]{{\bf [ FIXME: #1 ]}}
\newcommand{\wunits}[2]{\mbox{#1\,#2}}
\newcommand{\um}{\mbox{$\mu$m}}
\newcommand{\xum}[1]{\wunits{#1}{\um}}
\newcommand{\by}[2]{\mbox{#1$\times$#2}}
\newcommand{\byby}[3]{\mbox{#1$\times$#2$\times$#3}}


\newenvironment{tightlist}
{\begin{itemize}
 \setlength{\parsep}{0pt}
 \setlength{\itemsep}{-2pt}}
{\end{itemize}}

\newenvironment{titledtightlist}[1]
{\noindent
 ~~\textbf{#1}
 \begin{itemize}
 \setlength{\parsep}{0pt}
 \setlength{\itemsep}{-2pt}}
{\end{itemize}}

% Change spacing before and after section headers

\makeatletter
\renewcommand{\section}
{\@startsection {section}{1}{0pt}
 {-2ex}
 {1ex}
 {\bfseries\Large}}
\makeatother

\makeatletter
\renewcommand{\subsection}
{\@startsection {subsection}{1}{0pt}
 {-1ex}
 {0.5ex}
 {\bfseries\normalsize}}
\makeatother

% Reduce likelihood of a single line at the top/bottom of page

\clubpenalty=2000
\widowpenalty=2000

% Other commands and parameters

\pagestyle{myheadings}
\setlength{\parindent}{0in}
\setlength{\parskip}{10pt}

% Commands for register format figures.

\newcommand{\instbit}[1]{\mbox{\scriptsize #1}}
\newcommand{\instbitrange}[2]{\instbit{#1} \hfill \instbit{#2}}

\graphicspath{{./figs/}}


%-----------------------------------------------------------------------
% Document
%-----------------------------------------------------------------------

\begin{document}
\def\PYZsq{\textquotesingle}


\newcommand{\headertext}{EE142 Problem Set 9}
\renewcommand{\thesubsection}{\thesection.\alph{subsection}}

\title{\vspace{-0.4in}\Large \bf \headertext \vspace{-0.1in}}
\author{Vighnesh Iyer}

\date{\today}
\maketitle

\markboth{\headertext}{\headertext}
\thispagestyle{empty}

\section{Review of Important Concepts}
{\color{blue} Assume a memoryless distortion circuit is modeled by
$I_{out} = a_0 + a_1 V_{in} + a_2 V_{in}^2 + a_3 V_{in}^3$ and the input DC bias voltage is $V_{in,0}$.}

\begin{enumerate}[label=(\alph*)]
    \item {\color{blue} Derive IIP3, OIP3, $IP_{1dB}$, and $IP_{3dB}$}

    We begin by driving the circuit with a two-tone input with equal amplitude $A$ and frequencies $\omega_1$ and $\omega_2$:
    $$ S_i = A \cos{\left (\omega_1 t \right )} + A \cos{\left (\omega_2 t \right )} $$

    Now the full expanded form of the output can be derived:
    \begin{align*}
    S_o = \frac{9 a_{3}}{4} A^{3} \cos{\left (\omega_1 t \right )} + \frac{A^{3} a_{3}}{4} \cos{\left (3 \omega_1 t \right )} + \frac{9 a_{3}}{4} A^{3} \cos{\left (\omega_2 t \right )} + \frac{A^{3} a_{3}}{4} \cos{\left (3 \omega_2 t \right )} \\
        + \frac{3 a_{3}}{4} A^{3} \cos{\left (2 \omega_1 t + \omega_2 t \right )} + \frac{A^{2} a_{2}}{2} \cos{\left (2 \omega_1 t \right )} + \frac{A^{2} a_{2}}{2} \cos{\left (2 \omega_2 t \right )} + \\
        A^{2} a_{2} \cos{\left (\omega_1 t - \omega_2 t \right )} + A^{2} a_{2} \cos{\left (\omega_1 t + \omega_2 t \right )} + A^{2} a_{2} + A a_{1} \cos{\left (\omega_1 t \right )} + A a_{1} \cos{\left (\omega_2 t \right )}
    \end{align*}

    We define $IM3$ as $\frac{\text{Amplitude of one 3rd order IM product}}{\text{Amplitude of Fundamental}}$
    \begin{align*}
        IM3 &= \frac{3 a_3 / 4 \cdot A^3}{A a_1} \\
        &= \frac{3}{4} \frac{a_3}{a_1} A^2 \\
    \end{align*}

    To find IIP3, set $|IM3| = 1$ and solve for $A$. OIP3 is just the IIP3 power referenced to the output.
    \begin{align*}
        IIP3 &= \sqrt{\frac{4}{3} |\frac{a_1}{a_3}|} \\
        OIP3 &= IIP3 \cdot a_1
    \end{align*}

    $IP_{1dB}$ is defined by using a single-tone input and checking at what input power level the circuit's apparent gain has dropped by 1dB.

    \begin{align*}
        S_o = \frac{3 a_{3}}{4} A^{3} \cos{\left (\omega_1 t \right )} + \frac{A^{3} a_{3}}{4} \cos{\left (3 \omega_1 t \right )} &+ \frac{A^{2} a_{2}}{2} \cos{\left (2 \omega_1 t \right )} + \frac{A^{2} a_{2}}{2} + A a_{1} \cos{\left (\omega_1 t \right )} \\
        \text{Apparent Gain} &= \frac{a_1 A + \frac{3}{4} a_3 A^3}{A} \\
        &= a_1 (1 + \frac{3}{4} \frac{a_3}{a_1} A^2) \\
        20 \log(1 + \frac{3}{4} \frac{a_3}{a_1} A^2) &= -1 dB \\
        IP_{1dB} &= \sqrt{\frac{4}{3} |\frac{a_1}{a_3}|} \cdot \sqrt{0.11} \\
        IP_{3dB} &= \sqrt{\frac{4}{3} |\frac{a_1}{a_3}|} \cdot \sqrt{0.085}
    \end{align*}

    \item {\color{blue} If IIP3 is 10V, what is the input-blocker level that degrades the small-signal gain of the desired signal by 2dB?}

    We model an input signal $S_i = A \cos{\left (\omega_1 t \right )} + B \cos{\left (\omega_2 t \right )}$, where $\omega_1$ is the blocker, $\omega_2$ is the desired tone, and $A$ and $B$ are their magnitudes with $A >> B$.

    We want to look at the cubic terms in $S_o$, the output signal.
    \begin{align*}
        S_o \text{ contains }& \frac{3 a_{3}}{2} A^{2} B \cos{\left (\omega_2 t \right )} \\
        \text{Apparent Gain} &= \frac{a_1 B + a_3 \frac{3}{2} A^2 B}{B} \\
        &= a_1 (1 + \frac{3}{2} \frac{a_3}{a_1} A^2) \\
        &= a_1 (1 + \frac{2}{IIP3^2} A^2) \\
        20 \log(1 + \frac{2}{IIP3^2} A^2) &= -2 dB \\
        V_{blocker,max} &= 2.188 \text{ V}
    \end{align*}

    \item {\color{blue} Following part (b), what will be the tolerable blocker levels for a two-tone blocker?}

    We consider the following input signal: $ A \cos{\left (\omega_{d} t \right )} + B \cos{\left (\omega_{b1} t \right )} + B \cos{\left (\omega_{b2} t \right )} $

    Looking at the terms that contribute to $\omega_d$, we find the blocker term:
    \begin{align*}
        S_o \text{ contains } &= 3 A B^2 a_3
    \end{align*}
    this is just as expected since with two blocker tones, they will each contribute a blocking term to the desired signal.
    So, $V_{blocker,max} = 1.094 \text{ V}$

    \item {\color{blue} If IIP3 is 10V, what are the $IP_{1dB}$ for two-tone and three-tone input signals?}

    \begin{align*}
        IP_{1dB,one-tone} &= \sqrt{\frac{4}{3} |\frac{a_1}{a_3}|} \cdot \sqrt{0.11} = 3.32 \text{ V} \\
        IP_{1dB,two-tone} &= \sqrt{\frac{4}{9} |\frac{a_1}{a_3}|} \cdot \sqrt{0.11} = 1.9 \text{ V}\\
        IP_{1dB,three-tone} &= \sqrt{\frac{4}{15} |\frac{a_1}{a_3}|} \cdot \sqrt{0.11} = 1.48 \text{ V} \\
    \end{align*}

    \item {\color{blue} If the modeled circuit is a BJT with $I_{out} = I_s \exp(V_{be}/V_{T})$, use a math tool to find the actual output third-harmonic current as a function of the input magnitude. Compare the actual values to the estimated values via the power series.}

    I first thought of assigning $V_{be} = A \cdot \cos(2 \pi f t)$, and then finding the Fourier Series of $I_{out}$, but this turns out to be a non-elementary expression in terms of Bessel functions. I'm going to go with a numerical approach.

    I use Python to sample a 100 MHz cosine with amplitude $A$ at a 1 GHz sample rate. This is done for different amplitudes, and the sampled cosine serves as the $V_{be}$. This value is plugged into the $I_{out}$ exponential, and the FFT is taken to observe the power of the third harmonic.

    \begin{figure}[H]
        \centering \includegraphics[width=\textwidth]{problem1e_spectrum.pdf}
    \end{figure}

    It is clear that as the input amplitude goes down, the amplitude of the third harmonic goes down faster than the fundamental.
    Now, I sweep A and check $HD_3$:

    \begin{figure}[H]
        \centering \includegraphics[width=\textwidth]{problem1e_A_sweep.pdf}
    \end{figure}

    We know that with a power series expansion, $HD_3 = \frac{1}{4} \frac{a_3}{a_1} A^2$.
    For a BJT, $a_1 = I_Q/V_T$ and $a_3 = \frac{1}{6} (1/V_T)^3 I_Q$.

    \begin{figure}[H]
        \centering \includegraphics[width=\textwidth]{problem1e_HD3.pdf}
    \end{figure}

    So it's clear that the $HD_3$ formula only applies to small input signal amplitudes and is clearly off when ratios above 1.0 are considered since that is physically impossible.

\end{enumerate}

\section{Distortion of a Source Follower}
{\color{blue} For the source follower shown below, calculate the required bias current ($I_{bias}$ and W/L for the long-channel transistor to drive the load with a swing of 100 mV (at both $f_1$ and $f_2$), with IM3 equal to -50 dBc.}

{\color{red} Correction: vout= 0.1cos(2pi*f1*t)+0.1cos(2pi*f2*t) vin magnitude is not specified}

\begin{figure}[H]
    \centering \includegraphics[width=0.8\textwidth]{problem2_schematic.jpg}
\end{figure}

Notice that this is a source follower, or it can be represented as an amplifier with source degeneration in the AC path.
We use the feedback equations for the power series coefficients noting that the feedback factor $f$ is $R_L = 50 \Omega$ by the same reasoning used in the lecture slides for a BJT with emitter degeneration.
We assume the input magnitude is 120 mV arbitrarily.

\begin{align*}
    a_1 &= g_m \\
    a_2 &= \frac{1}{2} \mu C_{ox} \frac{W}{L} \\
    a_3 &= 0 \\
    b_1 = \frac{a_1}{1 + a_1 f} &= \frac{g_{m}}{R_{L} g_{m} + 1} \\
    b_2 = \frac{a_2}{(1 + a_1 f)^3} &= \frac{0.5 (W/L) \mu C_{ox}}{\left(R_{L} g_{m} + 1\right)^{3}} \\
    b_3 = \frac{a_3 (1 + a_1 f) - 2 a_2^2 f}{(1 + a_1 f)^5} &= - \frac{0.5 (W/L)^{2} R_{L} \mu C_{ox}^{2}}{\left(R_{L} g_{m} + 1\right)^{5}} \\
    IM3 = \frac{3}{4} \frac{b_3}{b_1} A^2 &= - \frac{0.375 (W/L)^{2} A^{2} R_{L} \mu C_{ox}^{2}}{g_{m} \left(R_{L} g_{m} + 1\right)^{4}}
\end{align*}

We have $IM3$ in terms of $\frac{W}{L}$ and $g_m$ (which is a function of $I_{bias}$). This is one equation.
The other equation concerns the swing of 100 mV required across $R_L$. The 3rd order intermod will cause gain compression.

\begin{align*}
    A_v &= \text{Apparent Current Gain} \cdot (\frac{1}{g_m} || r_o || R_L) = b_1(1 + \frac{9}{4} \frac{b_3}{b_1} A^2) \cdot (\frac{1}{g_m} || R_L) \\
    A_v &\geq \frac{100}{120}
\end{align*}

I tried this and I couldn't find a consistent set of equations. I found $g_m$ based on the usual source follower $A_V$ expression and substituted to find $W/L$.

The results seem wrong $g_m = 0.1$ and $W/L = 6000$. This would imply an $I_{bias} = 10$ mA for $V_{ov} = 0.2$.

\section{Pre-distortion and Source-degeneration Linearizer}
\begin{figure}[H]
    \centering \includegraphics[width=\textwidth]{problem3_schematic.jpg}
\end{figure}

\begin{enumerate}[label=(\alph*)]
    \item {\color{blue} For the above schematic, what are the OIP3 of the BJT stage for $R_e = 0 \Omega$ and $R_e = 0.02 \Omega$?}

    In general OIP3 [in dB] = IIP3 [in dB] + SS-gain [in dB].
    \begin{align*}
        IIP3 &= \sqrt{\frac{4}{3} |\frac{a_1}{a_3}|}
    \end{align*}

    For emitter degeneration, the equations were derived in class for the new BJT distortion coefficients.
    \begin{align*}
        IIP3_{no,degen} &= 0.147 \text{ A} \\
        IIP3_{0.02,degen} &= 0.627 \text{ A} \\
        OIP3_{no,degen} &= IIP3_{no,degen} \cdot g_m = 5.66 \\
        OIP3_{0.02,degen} &= IIP3_{0.02,degen} \cdot \frac{g_m}{1 + g_m R_E} = 13.64
    \end{align*}

    \item {\color{blue} What are the two possible $R_e$ for the BJT stage to have an OIP3 of 10A?}

    We set the equation for OIP3 to equal 10 and solve for $R_e = 0.0065 \Omega$.
    I'm unsure where the second solution for $R_e$ comes from.
\end{enumerate}

\end{document}
