%texexptitled======================================================================
% lab1-gcd
%-----------------------------------------------------------------------
%

\documentclass[11pt]{article}

% Package includes

\usepackage{graphicx}
\usepackage{color}
\usepackage{comment}
\usepackage{multirow}
\usepackage{askmaps}
\usepackage{amssymb}
\usepackage{amsmath}
\usepackage{tikz}
\usepackage{circuitikzgit}
\usetikzlibrary{arrows, positioning, shapes.geometric, circuits.logic.US}
\tikzstyle{line}=[draw]
\tikzstyle{arrow}=[draw, -latex]

% Wrap long URLs with hyphens
\PassOptionsToPackage{hyphens}{url}\usepackage{hyperref}
\usepackage{pdftexcmds}
\usepackage{upquote}
\usepackage{textcomp}
\usepackage{minted}
\usepackage[listings]{tcolorbox}
\usepackage{enumerate}
\usepackage{enumitem}
\usepackage{mathtools}
\DeclarePairedDelimiter{\ceil}{\Big\lceil}{\Big\rceil}

\tcbset{
texexp/.style={colframe=black, colback=lightgray!15,
         coltitle=white,
         fonttitle=\small\sffamily\bfseries, fontupper=\small, fontlower=\small},
     example/.style 2 args={texexp,
title={Question \thetcbcounter: #1},label={#2}},
}

\newtcolorbox{texexp}[1]{texexp}
\newtcolorbox[auto counter]{texexptitled}[3][]{%
example={#2}{#3},#1}

\setlength{\topmargin}{-0.5in}
\setlength{\textheight}{9in}
\setlength{\oddsidemargin}{0in}
\setlength{\evensidemargin}{0in}
\setlength{\textwidth}{6.5in}

% Useful macros

\newcommand{\note}[1]{{\bf [ NOTE: #1 ]}}
\newcommand{\fixme}[1]{{\bf [ FIXME: #1 ]}}
\newcommand{\wunits}[2]{\mbox{#1\,#2}}
\newcommand{\um}{\mbox{$\mu$m}}
\newcommand{\xum}[1]{\wunits{#1}{\um}}
\newcommand{\by}[2]{\mbox{#1$\times$#2}}
\newcommand{\byby}[3]{\mbox{#1$\times$#2$\times$#3}}


\newenvironment{tightlist}
{\begin{itemize}
 \setlength{\parsep}{0pt}
 \setlength{\itemsep}{-2pt}}
{\end{itemize}}

\newenvironment{titledtightlist}[1]
{\noindent
 ~~\textbf{#1}
 \begin{itemize}
 \setlength{\parsep}{0pt}
 \setlength{\itemsep}{-2pt}}
{\end{itemize}}

% Change spacing before and after section headers

\makeatletter
\renewcommand{\section}
{\@startsection {section}{1}{0pt}
 {-2ex}
 {1ex}
 {\bfseries\Large}}
\makeatother

\makeatletter
\renewcommand{\subsection}
{\@startsection {subsection}{1}{0pt}
 {-1ex}
 {0.5ex}
 {\bfseries\normalsize}}
\makeatother

% Reduce likelihood of a single line at the top/bottom of page

\clubpenalty=2000
\widowpenalty=2000

% Other commands and parameters

\pagestyle{myheadings}
\setlength{\parindent}{0in}
\setlength{\parskip}{10pt}

% Commands for register format figures.

\newcommand{\instbit}[1]{\mbox{\scriptsize #1}}
\newcommand{\instbitrange}[2]{\instbit{#1} \hfill \instbit{#2}}

\graphicspath{{./figs/}}


%-----------------------------------------------------------------------
% Document
%-----------------------------------------------------------------------

\begin{document}
\def\PYZsq{\textquotesingle}


\newcommand{\headertext}{EE142 Problem Set 8}
\renewcommand{\thesubsection}{\thesection.\alph{subsection}}

\title{\vspace{-0.4in}\Large \bf \headertext \vspace{-0.1in}}
\author{Vighnesh Iyer}

\date{\today}
\maketitle

\markboth{\headertext}{\headertext}
\thispagestyle{empty}

\section{System Analysis}
{\color{blue}A wireless receiver front-end is shown below:}
\begin{figure}[H]
    \centering \includegraphics[width=\textwidth-8cm]{problem1_schematic1.jpg}
\end{figure}

{\color{blue}We would like to receive the channel at 1800MHz, while we have additional chanels at 900MHz, 1805MHz, and 1810MHz.}
\begin{figure}[H]
    \centering \includegraphics[width=\textwidth-5cm]{problem1_schematic2.jpg}
\end{figure}

{\color{blue}The minimum detectable signal at 1800MHz is -100dBm.
The required signal to distortion ratio at the front-end output is 9dB.}

\begin{enumerate}[label=(\alph*)]
    \item {\color{blue}If the signal power at the 1810MHz channel is -33dBm, what is the maximum allowed power at the 1805MHz channel?}

    We can find the cascaded IIP3 of the front-end:
    \begin{align*}
        \frac{1}{IIP3^2} = \frac{1}{IIP3_A^2} + \frac{a_1^2}{IIP3_B^2}
    \end{align*}

    where the $IIP3$ terms in the above formula are voltages or currents.

    Here are some useful equations to convert power to voltage, and convert power gain to voltage gain:
    \begin{align*}
        \text{Power Gain in dB} &= 10 \log_{10}(\text{Power Gain in Linear Units}) \\
        \text{Power Gain in Linear Units} &= 10^{\text{Power Gain in dB} / 10} \\
        \text{Voltage Gain} &= \sqrt{\text{Power Gain in Linear Units}} \text{ assuming same } R_{in}, R_{out} \\
        \text{Power in dBm} &= 10 \log_{10}(\frac{\text{Power in Watts}}{10^{-3}}) \\
        \text{Voltage Induced} &= \sqrt{10^{-3} \cdot 10^{\text{Power in dBm}/10} \cdot 2R} \\
        \text{Power Delivered in dBm} &= 10 \cdot \log_{10}(\frac{V^2}{2R} / 10^{-3})
    \end{align*}

    We will assume operation in a $50\Omega$ environment.
    \begin{align*}
        IIP3_A &= 50 \text{ dBm} \rightarrow VIIP3_A = 100 \text{ V} \\
        IIP3_B &= 10 \text{ dBm} \rightarrow VIIP3_B = 1 \text{ V} \\
        a_1 &= 0.8912 \text{ V/V} \\
        VIIP_3 &= 1.1219 \text{ V} \\
        IIP_3 &= 11 \text{ dBm}
    \end{align*}

    As expected, the second stage's IIP3 dominates the cascaded IIP3.
    The power present at 1800MHz is caused by the intermodulation products of 1805MHz and 1810MHz:
    \begin{align*}
        V_{out,1800} = \frac{3 a_3}{4} A_{1805}^2 \cdot A_{1810}
    \end{align*}

    where $A_{x}$ is the voltage at $x$ MHz. We can find $a_3$ from IIP3:
    \begin{align*}
        IIP3 &= \sqrt{\frac{4}{3} \frac{|a_1|}{|a_3|}} \\
        a_3 &= 0.944 \\
        V_{out,1800} \leq V(-109 dBm) &\rightarrow P_{1805} \leq -26.5 \text{ dBm}
    \end{align*}

    \item {\color{blue}What is the required spec for the amplifier IIP2 if the signal power at the 900MHz channel can be as high as -30dBm?}
\end{enumerate}

\section{Distortion Analysis}
{\color{blue} In this problem you will do distortion analysis for a frequency-independent amplifier.}
\begin{figure}[H]
    \centering \includegraphics[width=\textwidth-5cm]{problem2_schematic.jpg}
\end{figure}

\begin{enumerate}[label=(\alph*)]
    \item {\color{blue} For the above amplifier, derive a power series to express the small-signal output voltage $v_{out}$ as a function of the small-signal input voltage $v_{in}$. Assume the transistors are long-channel devices.}


\end{enumerate}

\end{document}
