%texexptitled======================================================================
% lab1-gcd
%-----------------------------------------------------------------------
%

\documentclass[11pt]{article}

% Package includes

\usepackage{graphicx}
\usepackage{color}
\usepackage{comment}
\usepackage{multirow}
\usepackage{askmaps}
\usepackage{amssymb}
\usepackage{amsmath}
\usepackage{tikz}
\usepackage{circuitikzgit}
\usetikzlibrary{arrows, positioning, shapes.geometric, circuits.logic.US}
\tikzstyle{line}=[draw]
\tikzstyle{arrow}=[draw, -latex]

% Wrap long URLs with hyphens
\PassOptionsToPackage{hyphens}{url}\usepackage{hyperref}
\usepackage{pdftexcmds}
\usepackage{upquote}
\usepackage{textcomp}
\usepackage{minted}
\usepackage[listings]{tcolorbox}
\usepackage{enumerate}
\usepackage{enumitem}
\usepackage{mathtools}
\DeclarePairedDelimiter{\ceil}{\Big\lceil}{\Big\rceil}

\tcbset{
texexp/.style={colframe=black, colback=lightgray!15,
         coltitle=white,
         fonttitle=\small\sffamily\bfseries, fontupper=\small, fontlower=\small},
     example/.style 2 args={texexp,
title={Question \thetcbcounter: #1},label={#2}},
}

\newtcolorbox{texexp}[1]{texexp}
\newtcolorbox[auto counter]{texexptitled}[3][]{%
example={#2}{#3},#1}

\setlength{\topmargin}{-0.5in}
\setlength{\textheight}{9in}
\setlength{\oddsidemargin}{0in}
\setlength{\evensidemargin}{0in}
\setlength{\textwidth}{6.5in}

% Useful macros

\newcommand{\note}[1]{{\bf [ NOTE: #1 ]}}
\newcommand{\fixme}[1]{{\bf [ FIXME: #1 ]}}
\newcommand{\wunits}[2]{\mbox{#1\,#2}}
\newcommand{\um}{\mbox{$\mu$m}}
\newcommand{\xum}[1]{\wunits{#1}{\um}}
\newcommand{\by}[2]{\mbox{#1$\times$#2}}
\newcommand{\byby}[3]{\mbox{#1$\times$#2$\times$#3}}


\newenvironment{tightlist}
{\begin{itemize}
 \setlength{\parsep}{0pt}
 \setlength{\itemsep}{-2pt}}
{\end{itemize}}

\newenvironment{titledtightlist}[1]
{\noindent
 ~~\textbf{#1}
 \begin{itemize}
 \setlength{\parsep}{0pt}
 \setlength{\itemsep}{-2pt}}
{\end{itemize}}

% Change spacing before and after section headers

\makeatletter
\renewcommand{\section}
{\@startsection {section}{1}{0pt}
 {-2ex}
 {1ex}
 {\bfseries\Large}}
\makeatother

\makeatletter
\renewcommand{\subsection}
{\@startsection {subsection}{1}{0pt}
 {-1ex}
 {0.5ex}
 {\bfseries\normalsize}}
\makeatother

% Reduce likelihood of a single line at the top/bottom of page

\clubpenalty=2000
\widowpenalty=2000

% Other commands and parameters

\pagestyle{myheadings}
\setlength{\parindent}{0in}
\setlength{\parskip}{10pt}

% Commands for register format figures.

\newcommand{\instbit}[1]{\mbox{\scriptsize #1}}
\newcommand{\instbitrange}[2]{\instbit{#1} \hfill \instbit{#2}}

\graphicspath{{./figs/}}


%-----------------------------------------------------------------------
% Document
%-----------------------------------------------------------------------

\begin{document}
\def\PYZsq{\textquotesingle}


\newcommand{\headertext}{EE142 Problem Set 6}

\title{\vspace{-0.4in}\Large \bf \headertext \vspace{-0.1in}}
\author{Vighnesh Iyer}

\date{\today}
\maketitle

\markboth{\headertext}{\headertext}
\thispagestyle{empty}

\section{Design of a 5-Ghz Linear Microwave Amplifier}

{\color{blue}Use the 32nm PTM HP NMOS model with:

\begin{center}
\begin{tabular}{| l | l |} \hline
    $V_D$ & 0.9 V \\ \hline
    $V_G$ & 0.6 V \\ \hline
    $L$ & 32 nm \\ \hline
    $W_{tot}$ & 20 $\mu$m \\ \hline
    $N_{fingers}$ & 5 \\ \hline
\end{tabular}
\end{center}
}

\begin{figure}[H]
    \centering \includegraphics[width=\textwidth-6cm]{system_schematic.png}
\end{figure}

\subsection{Mixer Input Impedance}
{\color{blue}The mixer input is not $50 \Omega$. What will be the problem if this mixer is used as a connector module like the ones sold by MiniCircuits?}

The connector modules assume operation in a $50 \Omega$ environment.
Since the mixer input is $1000 \Omega$, any modules driving it will not be able to achieve the maximum possible power transfer to the mixer. Also, the reflections produced will distort the input waveform to the mixer.

\subsection{Stability Factor}
{\color{blue}Plot the stability factor ($k$), of this device versus frequency.
Is the device unconditionally stable at 1 Ghz?
If not, find a (passive) device load impedance such that the real part of the device
input impedance is negative at 1 Ghz.}

The stability factor ($k$) provides a necessary condition for unconditional stability of a 2-port network. It enforces that the input admittance/impedance of the two-port doesn't have a negative real part.

\begin{align*}
    Y_{in} &= y_{11} - \frac{y_{12}y_{21}}{y_{22} + Y_L} \\
    Y_{out} &= y_{22} - \frac{y_{12}y_{21}}{y_{11} + Y_S} \\
    \text{For instability: } \Re({Y_{in}}) < 0 &\text{ and } \Re(Y_{out}) < 0
\end{align*}

We set up a S-parameter simulation in ADS to calculate the stability factor versus frequency, using the provided transistor sizing and biasing.

\begin{figure}[H]
    \minipage{0.50\textwidth}
    \includegraphics[width=\linewidth]{partb_stability_schematic.pdf}
    \caption{Stability Factor Testbench}
    \endminipage\hfill
    \minipage{0.50\textwidth}
    \includegraphics[width=\linewidth]{partb_stability_factor_vs_frequency.pdf}
    \caption{Stability Factor vs. Frequency}
    \endminipage
\end{figure}

The transistor is not unconditionally stable at 1 Ghz. To find a passive load impedance that gives a negative real impedance, we can use the stability analysis (circle) feature of ADS.

\begin{figure}[H]
    \minipage{0.47\textwidth}
    \includegraphics[width=\linewidth]{partb_stability_circle.pdf}
    \caption{Stability Circle (not scaled to unit impedance circle)}
    \endminipage\hfill
    \minipage{0.47\textwidth}
    \includegraphics[width=\linewidth]{partb_stability_circle_scaled.pdf}
    \caption{Stability Circle (in unit impedance circle)}
    \endminipage
\end{figure}

It can be seen that for sufficiently inductive loads, the transistor isn't stable. For instance a load of $Z_L = 50 + 1.2j$ will lead to a negative input impedance.
\begin{figure}[H]
    \centering \includegraphics[width=\linewidth-6cm]{partb_yin_vs_zload.pdf}
    \caption{$Z_L = 50 + j Z_{load}$}
\end{figure}

\end{document}
