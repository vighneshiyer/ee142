%texexptitled======================================================================
% lab1-gcd
%-----------------------------------------------------------------------
%

\documentclass[11pt]{article}

% Package includes

\usepackage{graphicx}
\usepackage{color}
\usepackage{comment}
\usepackage{multirow}
\usepackage{askmaps}
\usepackage{amssymb}
\usepackage{amsmath}
\usepackage{tikz}
\usepackage{circuitikzgit}
\usetikzlibrary{arrows, positioning, shapes.geometric, circuits.logic.US}
\tikzstyle{line}=[draw]
\tikzstyle{arrow}=[draw, -latex]

% Wrap long URLs with hyphens
\PassOptionsToPackage{hyphens}{url}\usepackage{hyperref}
\usepackage{pdftexcmds}
\usepackage{upquote}
\usepackage{textcomp}
\usepackage{minted}
\usepackage[listings]{tcolorbox}
\usepackage{enumerate}
\usepackage{enumitem}
\usepackage{mathtools}
\DeclarePairedDelimiter{\ceil}{\Big\lceil}{\Big\rceil}

\tcbset{
texexp/.style={colframe=black, colback=lightgray!15,
         coltitle=white,
         fonttitle=\small\sffamily\bfseries, fontupper=\small, fontlower=\small},
     example/.style 2 args={texexp,
title={Question \thetcbcounter: #1},label={#2}},
}

\newtcolorbox{texexp}[1]{texexp}
\newtcolorbox[auto counter]{texexptitled}[3][]{%
example={#2}{#3},#1}

\setlength{\topmargin}{-0.5in}
\setlength{\textheight}{9in}
\setlength{\oddsidemargin}{0in}
\setlength{\evensidemargin}{0in}
\setlength{\textwidth}{6.5in}

% Useful macros

\newcommand{\note}[1]{{\bf [ NOTE: #1 ]}}
\newcommand{\fixme}[1]{{\bf [ FIXME: #1 ]}}
\newcommand{\wunits}[2]{\mbox{#1\,#2}}
\newcommand{\um}{\mbox{$\mu$m}}
\newcommand{\xum}[1]{\wunits{#1}{\um}}
\newcommand{\by}[2]{\mbox{#1$\times$#2}}
\newcommand{\byby}[3]{\mbox{#1$\times$#2$\times$#3}}


\newenvironment{tightlist}
{\begin{itemize}
 \setlength{\parsep}{0pt}
 \setlength{\itemsep}{-2pt}}
{\end{itemize}}

\newenvironment{titledtightlist}[1]
{\noindent
 ~~\textbf{#1}
 \begin{itemize}
 \setlength{\parsep}{0pt}
 \setlength{\itemsep}{-2pt}}
{\end{itemize}}

% Change spacing before and after section headers

\makeatletter
\renewcommand{\section}
{\@startsection {section}{1}{0pt}
 {-2ex}
 {1ex}
 {\bfseries\Large}}
\makeatother

\makeatletter
\renewcommand{\subsection}
{\@startsection {subsection}{1}{0pt}
 {-1ex}
 {0.5ex}
 {\bfseries\normalsize}}
\makeatother

% Reduce likelihood of a single line at the top/bottom of page

\clubpenalty=2000
\widowpenalty=2000

% Other commands and parameters

\pagestyle{myheadings}
\setlength{\parindent}{0in}
\setlength{\parskip}{10pt}

% Commands for register format figures.

\newcommand{\instbit}[1]{\mbox{\scriptsize #1}}
\newcommand{\instbitrange}[2]{\instbit{#1} \hfill \instbit{#2}}

\graphicspath{{./figs/}}


%-----------------------------------------------------------------------
% Document
%-----------------------------------------------------------------------

\begin{document}
\def\PYZsq{\textquotesingle}


\newcommand{\headertext}{EE142 Problem Set 4}

\title{\vspace{-0.4in}\Large \bf \headertext \vspace{-0.1in}}
\author{Vighnesh Iyer}

\date{\today}
\maketitle

\markboth{\headertext}{\headertext}
\thispagestyle{empty}

\section*{Problem 1}
Calculate the scattering parameters of the following circuits:

\begin{enumerate}[label=(\alph*)]
    \item \textcolor{blue}{Find the input $S_{11}$ for a general two-port terminated at port 2 with a load reflection coefficient of $\Gamma_L$.}

    We will call the wave going into port 1 $V_1^+$, the wave coming out of port 1 $V_1^-$, the wave \emph{into} port 2 $V_2^+$ and the wave out of port 2 $V_2^-$.

    We can then write the voltage waves in terms of the \emph{two-port} S parameters.
    \begin{align*}
        V_1^- &= S_{11} V_1^+ + S_{12} V_2^+ \\
        V_2^- &= S_{21} V_1^+ + S_{22} V_2^+
    \end{align*}

    Now, if port two is terminated by a load which results in reflection coefficient $\Gamma_L$, then the network effectively becomes a 1 port network. We can write the one-port $S_{11}$ in terms of the two-port S parameters.

    Throughout this problem we will assume that the two-port S parameters, the one port $S_{11}$, and $\Gamma_L$ are with respect to a reference of $Z_0$.
    \begin{align}
        V_2^+ &= V_2^- \Gamma_L \nonumber \\
        V_1^- &= S_{11} V_1^+ + S_{12} V_2^- \Gamma_L \\
        V_2^- &= S_{21} V_1^+ + S_{22} V_2^- \Gamma_L
    \end{align}

    \begin{align*}
        \text{Rewriting equ 2: } V_2^- &= \frac{S_{21} V_1^+}{1 - S_{22}\Gamma_L} \\
        \text{Plug into equ 1: } V_1^- &= S_{11}V_1^+ + S_{12}\Gamma_L \frac{S_{21} V_1^+}{1 - S_{22}\Gamma_L} \\
        \text{Finally: } \Aboxed{\frac{V_1^-}{V_1^+} &= S_{11} + \frac{S_{12}S_{21}\Gamma_L}{1 - S_{22}\Gamma_L}} \\
        \text{Notice: } \frac{V_1^-}{V_1^+} &= S_{11,one-port}
    \end{align*}

\item \textcolor{blue}{In the previous problem, what is the power that reaches the load in terms of the two-port scattering parameters and $\Gamma_L$? Suppose the input is driven with a matched source.}

    We derived the available power from the source for a 1-port network to be:

    $$ P_{avs} = \frac{|V_s|^2}{8 Z_0} $$

    where $V_s$ is the amplitude at the generator. Assuming a perfect input match:

    $$ V_1^+ = V_s / 2 $$

    The power seen by the load can be found as:

    $$ P_L = \frac{|V_2^-|^2 - |V_2^+|^2}{2 Z_0} = \frac{|V_2^-|^2(1-|\Gamma_L|^2)}{2 Z_0} $$

    Recall from the previous part that $V_2^-$ can be written in terms of S parameters:

    $$ V_2^- = \frac{S_{21} v_1^+}{1 - S_{22}\gamma_L} $$

    Then, $P_L$ can be written as such:

    $$ P_L =  \frac{|S_{21}|^2 |v_1^+|^2}{|1 - S_{22}\Gamma_L|^2} \cdot \frac{1 - |\Gamma_L|^2}{2 Z_0} $$

    Substituting for $V_1^+$:
    \begin{align*}
        \Aboxed{P_L = \frac{|S_{21}|^2 |v_s|^2}{4|1 - S_{22}\Gamma_L|^2} \cdot \frac{1 - |\Gamma_L|^2}{2 Z_0}}
    \end{align*}

    \item Derive the two-port scattering parameters of a three-port where port 3 is
        terminated in a load with reflection coefficient $\Gamma_L$.

\end{enumerate}
\end{document}
